%%%%%%%%%%%%%%%%%%%%%%%%%%%%%%%%% LAB-5 %%%%%%%%%%%%%%%%%%%%%%%%%%%%%%%%%%
%>>>>>>>>>>>>>>>>>>>>>>>>>> ПЕРЕМЕННЫЕ >>>>>>>>>>>>>>>>>>>>>>>>>>>>>>>>>>>
%>>>>> Информация о кафедре
%\newcommand{\year}{2021 г.}  % Год устанавливается автоматически
\newcommand{\city}{Санкт-Петербург}  %  Футер, нижний колонтитул на титульном листе
\newcommand{\university}{Национальный исследовательский университет ИТМО}  % первая строка
\newcommand{\department}{Факультет программной инженерии и компьютерной техники}  % Вторая строка
\newcommand{\major}{Направление системного и прикладного программного обеспечения}  % Треьтя строка
%<<<<< Информация о кафедре

%>>>>> Назание работы
\newcommand{\reporttype}{ОТЧЕТ ПО ЛАБОРАТОРНОЙ РАБОТЕ} % тип работы, (главный заголовок титульного листа)
\newcommand{\lab}{Лабораторная работа}          % вид работы
\newcommand{\labnumber}{№ 1}                    % порядковый номер работы
\newcommand{\subject}{Вычислительная математика}         % учебный предмет
\newcommand{\labtheme}{Решение системы линейных алгебраических уравнений СЛАУ}  % Тема лабораторной работы
\newcommand{\variant}{№ 2 (№ 22 в списке группы)}                % номер варианта работы

\newcommand{\student}{Тюрин Иван Николаевич}    % определение ФИО студента
\newcommand{\studygroup}{P32131}                 % определение учебной группы 
\newcommand{\teacher}{Малышева Т. А.,\\[1mm]     % ФИО лектора
                        Бострикова Д. К.}          % ФИО практика
%<<<<<<<<<<<<<<<<<<<<<<<<<< ПЕРЕМЕННЫЕ <<<<<<<<<<<<<<<<<<<<<<<<<<<<<<<<<<<


%>>>>>>>>>>>>>>>>>>>>>> ПРЕАМБУЛА >>>>>>>>>>>>>>>>>>>>>>>>>
\input{preamble}
%<<<<<<<<<<<<<<<<<<<<<< ПРЕАМБУЛА <<<<<<<<<<<<<<<<<<<<<<<<<



%%%%%%%%%%%%%%%%%%% СОДЕРЖИМОЕ ОТЧЕТА %%%%%%%%%%%%%%%%%%%%%
%>>>>>>>>>>>>>>> ''''''''''''''''''''''' >>>>>>>>>>>>>>>>>>
\begin{document}


%>>>>>>>>>>>>>>>> ОПРЕДЕЛЕНИЕ НАЗВАНИЙ >>>>>>>>>>>>>>>>>>>>
% Переоформление некоторых стандартных названий
%\renewcommand{\chaptername}{Лабораторная работа}
\renewcommand{\chaptername}{\lab\ \labnumber} % переименование глав
\def\contentsname{Содержание} % переименование оглавления
%<<<<<<<<<<<<<<<< ОПРЕДЕЛЕНИЕ НАЗВАНИЙ <<<<<<<<<<<<<<<<<<<<
% \setlength{\itemsep}{0pt} % установка расстояния между строчками в списках можно использовать локально внутри списка списке
% \setlength{\parskip}{0pt} % 
% \setlength{\parsep}{0pt}  % 

%>>>>>>>>>>>>>>>>> ТИТУЛЬНАЯ СТРАНИЦА >>>>>>>>>>>>>>>>>>>>>
\include{titlepage}
%<<<<<<<<<<<<<<<<< ТИТУЛЬНАЯ СТРАНИЦА <<<<<<<<<<<<<<<<<<<<<


%>>>>>>>>>>>>>>>>>>>>> СОДЕРЖАНИЕ >>>>>>>>>>>>>>>>>>>>>>>>>
% Содержание
\tableofcontents
%<<<<<<<<<<<<<<<<<<<<< СОДЕРЖАНИЕ <<<<<<<<<<<<<<<<<<<<<<<<<


%%%%%%%%%%%%%%%%%%%%%%% КОД РАБОТЫ %%%%%%%%%%%%%%%%%%%%%%%%
%>>>>>>>>>>>>>>>>>>>'''''''''''''''''>>>>>>>>>>>>>>>>>>>>>
\newpage
\Chapter{\lab\ \labnumber}{\labtheme}{}

\Section{Задание варианта \variant}
\begin{center}
, , ,
\\[2mm]
\textit{ Метод Гаусса с выбором главного элемента по столбцам }
\\[2mm]
' ' '
\end{center}

\Section{Цель работы}
Изучить способы численных методов решения системы линейных алгебраических уравнений и реализовать один из них.

\Section{Описание метода, расчетные формулы}
Метод Гаусса с выбором главного элемента по столбцам --- прямой метод решения СЛАУ. Метод состоит из двух этапов: \begin{enumerate}
    \item подготовка матрицы (<<прямой ход>>),
    \item вычисление вектора переменных (<<обратный ход>>).
\end{enumerate}
Во время первого этапа происходит приведение данной квадратной матрицы к треугольному виду с помощью последовательного исключения переменных из всех нижележащих строчек. Математически шаг с исключением переменной $x_i$ из уравнений начиная с $i+1$ можно описать, как
\begin{equation}
    a'_{kj} = a_{kj}-\cf{a_{ki}}{a_{ii}}a_{kj},\; \forall i,j\in[1..n],\, k\in[(i+1)..n].
\end{equation}
При этом модификация \textit{выбора главного элемента по столбцам} заключается в том, что перед началом исключения мы, если это нужно, перестанавливаем две строчки в матрице так, чтобы главный элемент в текущей строке был как можно большим по абсолютному значению, т.е. $a_{ii}\geq a_{ji} \;\forall i\in[1..n],\,j\in[(i+1)..n]$. Этот действие позволяет повысить точность вычислений на компьютере, т.к. таким образом мы стараемся избавиться от погрешности от деления на близкое к нулю число. В добавок к этому, посчитав количество перестановок строчек, мы можем установить знак определителя нашей изначальной матрицы (его значение, с точностью до знака, мы можем вычислить перемножив элементы на диагонали полученной треугольной матрицы), а именно \begin{equation}
    \det A = (-1)^k\prod\limits^n_{i=1}a_{ii},
\end{equation} где $k$ --- число перестановок строк (или столбцов) матрицы при ее приведении к треугольному виду в соответствии с модификацией.

\Subsection{Листинг программы}
Основную часть программной реализации на языке программировани Kotlin можно посмотреть в листинге \ref{lst:core}. Весь код представлен в личном репозитории \cite{itmocompmath}.
\lstinputlisting[caption={Реализация на языке программирования Kolin основной логики решания СЛАУ},label={lst:core}]{./res/core-listing.txt}

\Section{Примеры и результаты работы программы}
В утилите реализована возможность ввода данных через файл специального формата даннных, заполнение матрицы случайными числами и режим интерактивного ввода.

\Subsection{Листинг тестов}
Для проверки работоспособности программы были написаны тесты, их содержимое представлено в листинге \ref{lst:test}. 
\lstinputlisting[caption={Реализация на языке программирования Kolin тестов утилиты для решения СЛАУ},label={lst:test}]{./res/test-listing.txt}

\Subsection{Листинг результатов работы тестов}
Соответственно вывод резултатов тестов представлен в листинге \ref{lst:testoutput}. Текстовый файл использовавшийся при тестировании так же можно найти среди ресурсов для тестов в личном репозитории \cite{itmocompmath}.
\lstinputlisting[caption={Вывод с тандартный поток тестов утилиты для решения СЛАУ},label={lst:testoutput}]{./res/test-output-listing.txt}

\Section{Вывод}
В ходе выполнения данной лабораторной работы углубили понимание работы методов решения СЛАУ, реализовали на языке Kotlin требуемую утилиту для их решения. \\
При решении СЛАУ использовался метод Гаусса с выбором наибольшего элемента по строкам, обычный метод Гаусса оказалось достаточно легко реализовать и потом не трудно модифициоровать до метода с выбором максимального элемента. Погрешности вычисления которые возникают при работе его реализации достаточно малы при не очень больших по модулю значениях, в противном случае возникает ошибка представления дробных чисел в копьютере: не хватает точности для больших по модулю чисел. Эта погрешность возникает потому что требуется перемножать много чисел во время расчетов.

\newpage
%<<<<<<<<<<<<<<<<<<<<<< КОД РАБОТЫ <<<<<<<<<<<<<<<<<<<<<<<<

%>>>>>>>>>>>>>>>> СПИСОК ЛИТЕРАТУРЫ >>>>>>>>>>>>>>>>>>>>>>>
\begin{thebibliography}{}
\bibitem{itmocompmath} Cсылка на личный репозиторий GitHub: \url{https://github.com/e1turin/itmo-comp-math/tree/main/lab-1}\\
\end{thebibliography}  % Для соответсвия гост, придется доработать. Нужен файл .bib
%<<<<<<<<<<<<<<<<<<<< СПИСОК ЛИТЕРАТУРЫ <<<<<<<<<<<<<<<<<<<

\end{document}
%<<<<<<<<<<<<<<<< ,,,,,,,,,,,,,,,,,,,,,,, <<<<<<<<<<<<<<<<<
%<<<<<<<<<<<<<<<<<<< СОДЕРЖИМОЕ ОТЧЕТА <<<<<<<<<<<<<<<<<<<<
