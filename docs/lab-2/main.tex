%%%%%%%%%%%%%%%%%%%%%%%%%%%%%%%%% LAB-5 %%%%%%%%%%%%%%%%%%%%%%%%%%%%%%%%%%
%>>>>>>>>>>>>>>>>>>>>>>>>>> ПЕРЕМЕННЫЕ >>>>>>>>>>>>>>>>>>>>>>>>>>>>>>>>>>>
%>>>>> Информация о кафедре
%\newcommand{\year}{2021 г.}  % Год устанавливается автоматически
\newcommand{\city}{Санкт-Петербург}  %  Футер, нижний колонтитул на титульном листе
\newcommand{\university}{Национальный исследовательский университет ИТМО}  % первая строка
\newcommand{\department}{Факультет программной инженерии и компьютерной техники}  % Вторая строка
\newcommand{\major}{Направление системного и прикладного программного обеспечения}  % Треьтя строка
%<<<<< Информация о кафедре

%>>>>> Назание работы
\newcommand{\reporttype}{ОТЧЕТ ПО ЛАБОРАТОРНОЙ РАБОТЕ} % тип работы, (главный заголовок титульного листа)
\newcommand{\lab}{Лабораторная работа}          % вид работы
\newcommand{\labnumber}{№ 2}                    % порядковый номер работы
\newcommand{\subject}{Вычислительная математика}         % учебный предмет
\newcommand{\labtheme}{Решение нелинейных уравнений и систем нелинейных уравнений}  % Тема лабораторной работы
\newcommand{\variant}{№ 21}                % номер варианта работы

\newcommand{\student}{Тюрин Иван Николаевич}    % определение ФИО студента
\newcommand{\studygroup}{P32131}                 % определение учебной группы 
\newcommand{\teacher}{Малышева Т. А.,\\[1mm]     % ФИО лектора
                        Бострикова Д. К.}          % ФИО практика
%<<<<<<<<<<<<<<<<<<<<<<<<<< ПЕРЕМЕННЫЕ <<<<<<<<<<<<<<<<<<<<<<<<<<<<<<<<<<<


%>>>>>>>>>>>>>>>>>>>>>> ПРЕАМБУЛА >>>>>>>>>>>>>>>>>>>>>>>>>
\input{preamble}
%<<<<<<<<<<<<<<<<<<<<<< ПРЕАМБУЛА <<<<<<<<<<<<<<<<<<<<<<<<<



%%%%%%%%%%%%%%%%%%% СОДЕРЖИМОЕ ОТЧЕТА %%%%%%%%%%%%%%%%%%%%%
%>>>>>>>>>>>>>>> ''''''''''''''''''''''' >>>>>>>>>>>>>>>>>>
\begin{document}


%>>>>>>>>>>>>>>>> ОПРЕДЕЛЕНИЕ НАЗВАНИЙ >>>>>>>>>>>>>>>>>>>>
% Переоформление некоторых стандартных названий
%\renewcommand{\chaptername}{Лабораторная работа}
\renewcommand{\chaptername}{\lab\ \labnumber} % переименование глав
\def\contentsname{Содержание} % переименование оглавления
%<<<<<<<<<<<<<<<< ОПРЕДЕЛЕНИЕ НАЗВАНИЙ <<<<<<<<<<<<<<<<<<<<
% \setlength{\itemsep}{0pt} % установка расстояния между строчками в списках можно использовать локально внутри списка списке
% \setlength{\parskip}{0pt} % 
% \setlength{\parsep}{0pt}  % 

%>>>>>>>>>>>>>>>>> ТИТУЛЬНАЯ СТРАНИЦА >>>>>>>>>>>>>>>>>>>>>
\include{titlepage}
%<<<<<<<<<<<<<<<<< ТИТУЛЬНАЯ СТРАНИЦА <<<<<<<<<<<<<<<<<<<<<


%>>>>>>>>>>>>>>>>>>>>> СОДЕРЖАНИЕ >>>>>>>>>>>>>>>>>>>>>>>>>
% Содержание
\tableofcontents
%<<<<<<<<<<<<<<<<<<<<< СОДЕРЖАНИЕ <<<<<<<<<<<<<<<<<<<<<<<<<


%%%%%%%%%%%%%%%%%%%%%%% КОД РАБОТЫ %%%%%%%%%%%%%%%%%%%%%%%%
%>>>>>>>>>>>>>>>>>>>'''''''''''''''''>>>>>>>>>>>>>>>>>>>>>
\newpage
\Chapter{\lab\ \labnumber}{\labtheme}{}

\Section{Задание варианта \variant}
\begin{center}
, , ,
\end{center}
\textit{%
    Для вычислительной части: метод хорд (1), метод половинного деления (2), метод простой итерации (3) и функция
}
$$1,8 \cdot x^3 - 2,47 \cdot x^2 - 5,53 \cdot x + 1,539.$$
\textit{%
    Для программной части: метод ньютона, метод хорд, метод простой итерации для уравнения и системы уравнений.
}
\begin{center}
' ' '
\end{center}

\Section{Цель работы}
Изучить численные методы решения нелинейных уравнений и их систем, найти корни заданного нелинейного равнения/системы нелинейных уравнений, выполнить программную реализацию методов.

\Section{Вычислительная часть работы}
\Subsection{Подготовка к вычислениям}
Отсделилили корни заданого нелинейного уравнения графически с помощью программы Desmos. График функции на интервале $[-5;5]$ можно видеть на рисунке \ref{fig:graph}.
\begin{figure}
    \centering
    \boxed{
        \includegraphics[width=\textwidth]{res/Screenshot-manual-function-graph.png}
    }
    \caption{График данной функции на промежутке $[-5; 5]$}
    \label{fig:graph}
\end{figure}

Получили следующие результаты: \begin{itemize}
    \item первый корень $x_1 \in [-1,4;\; -1,3]$,
    \item второй корень $x_2 \in [0,2;\; 0,3]$,
    \item третий корень $x_3 \in [2,4;\; 2,5]$.
\end{itemize}
\Subsection{Вычисления}
Используя табличный процессор Google Sheets вычислили приближенные значения корней уравнения.
\begin{enumerate}
    \item Вычисление первого корня методом хорд можно видеть в табдице \ref{tab:first-root}.
    \item Вычисление второго корня методом половинного деления можно видеть в табдице \ref{tab:second-root}.
    \item Вычисление третьего корня методом простой итерации можно видеть в табдице \ref{tab:third-root}.
\end{enumerate}
Важно при этом уточнить, что при вычислении третьего корня методом простой итерации была использована функция приближения полученная после выражения переменной из старшего члема многочлена: $x_{i+1} = \sqrt[3]{(2,47 \cdot x_i^2 + 5,53 \cdot x_i + 1,539)/1,8}$.
\begin{table}[]
    \centering
    \begin{tabular}{|c|c|c|c|c|c|c|c|}\hline
$k$	     &	$a$	     &	$b$	     &	$x$	     &	$f(a)$	 &	$f(b)$	 &	$f(x)$	 &	$|x_{k+1} - x_k|$\\\hline
1,0000	 &	-1,4000	 &	-1,3000	 &	-1,3545	 &	-0,4994	 &	0,5991	 &	0,0242	 &	-    \\\hline
2,0000	 &	-1,4000	 &	-1,3545	 &	-1,3566	 &	-0,4994	 &	0,0242	 &	0,0009	 &	0,00210	\\\hline
3,0000	 &	-1,4000	 &	-1,3566	 &	-1,3567	 &	-0,4994	 &	0,0009	 &	0,0000	 &	0,00008	\\\hline
4,0000	 &	-1,4000	 &	-1,3567	 &	-1,3567	 &	-0,4994	 &	0,0000	 &	0,0000	 &	0,00000	\\\hline
    \end{tabular}
    \caption{Нахождение первого корня уравнения методом хорд}
    \label{tab:first-root}
\end{table}
\begin{table}[]
    \centering
    \begin{tabular}{|c|c|c|c|c|c|c|c|}\hline
$k$	 &	$a$	 &	$b$	 &	$x$	 &	$f(a)$	 &	$f(b)$	 &	$f(x)$	 &	$|a-b|$	\\\hline
1,0000	 &	0,2000	 &	0,3000	 &	0,2500	 &	0,3486	 &	-0,2937	 &	0,0302	 &	0,1000	\\\hline
2,0000	 &	0,2500	 &	0,3000	 &	0,2750	 &	0,0302	 &	-0,2937	 &	-0,1311	 &	0,0500	\\\hline
3,0000	 &	0,2500	 &	0,2750	 &	0,2625	 &	0,0302	 &	-0,1311	 &	-0,0503	 &	0,0250	\\\hline
4,0000	 &	0,2500	 &	0,2625	 &	0,2563	 &	0,0302	 &	-0,0503	 &	-0,0100	 &	0,0125	\\\hline
5,0000	 &	0,2500	 &	0,2563	 &	0,2531	 &	0,0302	 &	-0,0100	 &	0,0102	 &	0,0062	\\\hline
    \end{tabular}
    \caption{Нахождение второго корня уравнения методом половинного деления}
    \label{tab:second-root}
\end{table}
\begin{table}[]
    \centering
    \begin{tabular}{|c|c|c|c|c|}\hline
$k$	 &	$x_k$	         & 	$x_{k+1}=\sqrt[3]{...}$	  &	$f(x_{k+1})$	 &	$|x_{k+1} - x_k|$	 \\\hline	
1	 &	-2	             &	-1,147393101	          &	1,913301856	     &	0,8526068992	     \\\hline	
2	 &	-1,147393101	 &	-1,370381379	          &	-0,1536172866	 &	0,2229882787	     \\\hline	
3	 &	-1,370381379	 &	-1,355062491	          &	0,01841095071	 &	0,01531888879	     \\\hline	
4	 &	-1,355062491	 &	-1,356916747	          &	-0,002166862467	 &	0,001854255854	     \\\hline	
    \end{tabular}
    \caption{Нахождение третьего корня уравнения методом простой итерации}
    \label{tab:third-root}
\end{table}

\Section{Программная часть работы}
\Subsection{Листинг программы}
Основную часть программной реализации на языке программировани Kotlin можно посмотреть в листинге \ref{lst:core}. Весь код представлен в личном репозитории \cite{itmocompmath}.
\lstinputlisting[caption={Реализация на языке программирования Kolin основной логики методов решения нелинейных уравнений},label={lst:core}]{./res/core-listing.txt}

\Subsection{Примеры и результаты работы программы}
В утилите реализована возможность ввода данных через файл специального формата даннных CON --- производной от JSON c отличием, что в нем все запятые заменены на точки с запятыми и все точки заменены на запятые; нужен он для удовлетворения дополнительным требованиям практика по использованию запятых в качестве разделителя при вводе и выводе данных \cite{wikicommas}. Внешний вид пользовательского приложения можно увидеть на скриншотaх \begin{itemize}
    \item пользовательский интерфейс при отображении решения уравнения \ref{fig:screenshot-solution},
    \item при отображении ошибки \ref{fig:screenshot-error},
    \item при отображении решения уравнения системы уравнений \ref{fig:screenshot-system},
\end{itemize}.

\begin{figure}
    \centering
    \boxed{
        \includegraphics[width=\textwidth]{res/screenshot-example-solution.png}
    }
    \caption{Пользовательский интейфейс при решении уравнения}
    \label{fig:screenshot-solution}
\end{figure}

\begin{figure}
    \centering
    \boxed{
        \includegraphics[width=\textwidth]{res/screenshot-example-error.png}
    }
    \caption{Пользовательский интейфейс при возникновении ошибки}
    \label{fig:screenshot-error}
\end{figure}

\begin{figure}
    \centering
    \boxed{
        \includegraphics[width=\textwidth]{res/screenshot-example-system.png}
    }
    \caption{Пользовательский интейфейс при решении системы уравнений}
    \label{fig:screenshot-system}
\end{figure}

\Section{Вывод}
В ходе выполнения данной лабораторной работы углубили понимание работы методов решения нелинейных уравнений, реализовали на языке Kotlin требуемое приложение с графическим интерфейсом для их решения и самостоятельно вычислити решение уравнения.

Выяснили на практике, что разные методы по разному точны и имеют определенные сферы применения. Реализовать базовые методы решения нелинейных уравнений оказалось довольно легко. Так же поняли, что реализовать графическое приложение гораздо сложнее, чем реализовать аналогичное консольное и что это задание не соответствует направленности курса <<Вычислительная математика>>. Получили большой опыт в создании десктомных приложений с помощью Compose Multiplatform на языке программирования Kotlin.

\newpage
%<<<<<<<<<<<<<<<<<<<<<< КОД РАБОТЫ <<<<<<<<<<<<<<<<<<<<<<<<

%>>>>>>>>>>>>>>>> СПИСОК ЛИТЕРАТУРЫ >>>>>>>>>>>>>>>>>>>>>>>
\begin{thebibliography}{}
\bibitem{itmocompmath} Cсылка на личный репозиторий GitHub: \url{https://github.com/e1turin/itmo-comp-math/tree/main/lab-1}\\
\end{thebibliography}  % Для соответсвия гост, придется доработать. Нужен файл .bib
%<<<<<<<<<<<<<<<<<<<< СПИСОК ЛИТЕРАТУРЫ <<<<<<<<<<<<<<<<<<<

\end{document}
%<<<<<<<<<<<<<<<< ,,,,,,,,,,,,,,,,,,,,,,, <<<<<<<<<<<<<<<<<
%<<<<<<<<<<<<<<<<<<< СОДЕРЖИМОЕ ОТЧЕТА <<<<<<<<<<<<<<<<<<<<
